%! Author = thomas del moro
%! Date = 31/10/2023

\documentclass{article}

% Language setting
% Replace `english' with e.g. `spanish' to change the document language
\usepackage[italian]{babel}

% Set page size and margins
% Replace `letterpaper' with `a4paper' for UK/EU standard size
\usepackage[letterpaper,top=2cm,bottom=2cm,left=3cm,right=3cm,marginparwidth=1.75cm]{geometry}

% Useful packages
\usepackage{amsmath}
\usepackage{amsfonts}
\usepackage{graphicx}
\usepackage{mathrsfs}
\usepackage[colorlinks=true, allcolors=blue]{hyperref}
\usepackage{subcaption}
\usepackage{float}
\usepackage{wrapfig}

\usepackage[font=small,labelfont=bf]{caption}

\title{}
\author{Thomas Del Moro}

\begin{document}
    \maketitle
    \begin{abstract}
        Questo studio esplora l'applicazione di vari modelli di epidemie nell'ambito della fisica statistica e dei
        sistemi complessi, in particolare per quanto riguarda la diffusione di malattie infettive in una popolazione o
        la percolazione di informazioni in una rete sociale.
        Attraverso l'applicazione di questi modelli, esaminiamo fattori come la densità di popolazione, i contatti fra
        gli individui e le strategie di mitigazione basate sulla percezione del rischio.
        Le simulazioni illustrano come tali modelli possano fornire una prospettiva efficace per comprendere e gestire
        fenomeni epidemici complessi.
    \end{abstract}


    \section{Introduzione}
    La modellazione delle epidemie è un'importante area di ricerca che utilizza concetti della fisica statistica e dei
    sistemi complessi per comprendere e prevedere la diffusione delle malattie infettive.
    I modelli epidemiologici forniscono un quadro teorico per analizzare come le malattie si diffondono all'interno di
    una popolazione e come le strategie di controllo possono influenzare questa diffusione.
    Tra i modelli più conosciuti vi è il modello SIR (Susceptible-Infectious-Recovered), che suddivide la popolazione
    in tre compartimenti: suscettibili, infetti e guariti.\\
    Questi modelli, tuttavia, spesso non tengono conto della complessità delle interazioni sociali e della struttura di
    rete all'interno delle popolazioni umane.
    È però possibile, utilizzando metodi ispirati alla fisica, come la teoria
    dei grafi e la dinamica dei sistemi fuori dall'equilibrio, sviluppare modelli più sofisticati che catturino meglio
    la natura intricata delle interazioni umane.\\
    In questa revisione, esploreremo l'applicazione di vari modelli di epidemie e, attraverso simulazioni computerizzate,
    esamineremo come questi modelli possano essere utilizzati per comprendere e gestire fenomeni epidemici complessi.


    \section{Forest Fire}
    Il modello più semplice di un'epidemia trascura completamente la dinamica dell'agente virale. Se supponiamo inoltre
    che gli ospiti non si muovano e siano disposti su una griglia regolare con una certa densità.
    Otteniamo in questo modo il modello di diffusione di un incendio boschivo senza vento (Forest Fire).\\
    Ogni cella della griglia può essere in uno di tre stati: vuota, occupata da un albero sano oppure occupata da un albero
    incendiato. Per semplicità abbiamo trascurato il caso di albero già bruciato (immune), che vedremo nella prossima sezione.\\
    Il processo di diffusione segue una curva a S, più o meno ripida e lunga a seconda del parametro di densità.
    In particolare, se la densità è troppo bassa, l'incendio si estinguerà rapidamente, mentre se questa è alta l'incendio
    di propagherà a tutto il reticolo.
    \begin{figure}[H]
        \centering
        \begin{subfigure}[b]{0.49\linewidth}
         \centering
         \includegraphics[width=\linewidth]{../images/forestfire_single05image}
     \end{subfigure}
        \hfill
        \begin{subfigure}[b]{0.49\linewidth}
         \centering
         \includegraphics[width=\linewidth]{../images/forestfire_single05}
     \end{subfigure}
     \hfill
     \begin{subfigure}[b]{0.49\linewidth}
         \centering
         \includegraphics[width=\linewidth]{../images/forestfire_single07image}
     \end{subfigure}
        \hfill
        \begin{subfigure}[b]{0.49\linewidth}
         \centering
         \includegraphics[width=\linewidth]{../images/forestfire_single07}
     \end{subfigure}
     \caption{Simulazioni del modello Forest Fire con densità 0.5 (in alto) e 0.7 (in basso). Sulla sinistra si possono
     osservare le immagini del reticolo, mentre sulla destra la curva di diffusione dell'incendio.}
     \label{fig:forestfire}
    \end{figure}
    Notiamo facilmente che se la densità è relativamente bassa, è probabile che l'incendio di estingua velocemente.
    Possiamo quindi osservare il numero finale di alberi incendiati al variare della densità del reticolo. I risultati,
    mostrati nel grafico in Figura \ref{fig:forestfire_density}, evidenziano di nuovo una curva a S, con un punto di
    transizione tra un regime in cui l'incendio si estingue rapidamente e uno in cui si propaga a tutto il reticolo.
    \begin{figure}[H]
        \centering
        \includegraphics[width=0.6\linewidth]{../images/forestfire_final}
        \caption{Numero finale di alberi incendiati al variare della densità del reticolo.}
        \label{fig:forestfire_density}
    \end{figure}



    \section{SIR}
    Il modello SIR rappresenta uno dei più conosciuti e più semplici modelli epidemiologici.
    È in parte analogo al modello Forest Fire, ma con una dinamica probabilistica, leggermente più complessa di quella vista sopra.\\
    Supponiamo di avere una popolazione di $N$ individui disposti su una griglia. In questo caso tutte le celle sono
    occupate da un individuo, ma questo, a ogni istante di tempo, sarà infettato dagli individui vicini solo con una certa probabilità $p$.
    Le celle possono dunque essere suddivise nelle tre categorie:
    \begin{itemize}
        \item \textbf{Suscettibili (S)}: individui sani che possono essere infettati.
        \item \textbf{Infetti (I)}: individui infetti che possono infettare gli individui sani.
        \item \textbf{Refrattari (R)}: individui infettati e guariti che non possono più ammalarsi (immuni).
    \end{itemize}
    Per eseguire le simulazioni possiamo inizializzare il reticolo con un numero arbitrario di infetti e osservare come la malattia si diffonde a
    seconda dei parametri di infettività e di guarigione, fino a che non sarà più in grado di infettare nuovi individui.
    \begin{figure}[H]
        \centering
        \begin{subfigure}[b]{0.49\linewidth}
         \centering
         \includegraphics[width=\linewidth]{../images/sir_image1}
     \end{subfigure}
        \hfill
        \begin{subfigure}[b]{0.49\linewidth}
         \centering
         \includegraphics[width=\linewidth]{../images/sir_image20}
     \end{subfigure}
     \caption{Simulazioni del modello SIR rispettivamente a partire da 1 infetto (sinistra) e da 20 infetti (destra).}
     \label{fig:sir_image}
    \end{figure}
    Le curve caratteristiche del modello SIR possono essere ottenute applicando le assunzioni di campo medio, ovvero
    eliminando le correlazioni spaziali tra gli individui. In questa relazione ci limiteremo a confrontare le curve teoriche
    con quelle empiriche ottenute dalle simulazioni sopra descritte. Tale indagine è mostrata in Figura \ref{fig:sir_curves}.
    \begin{figure}[H]
        \centering
        \begin{subfigure}[b]{0.49\linewidth}
         \centering
         \includegraphics[width=\linewidth]{../images/sir_equations2}
            \caption{Curve teoriche del modello SIR}
        \end{subfigure}
        \hfill
        \begin{subfigure}[b]{0.49\linewidth}
         \centering
         \includegraphics[width=\linewidth]{../images/sir_curves1}
            \caption{Curve empiriche di una simulazione del modello SIR}
        \end{subfigure}
        \caption{Curve caratteristiche teoriche ed empiriche del modello SIR.}
        \label{fig:sir_curves}
    \end{figure}

    Il fattore più critico per la diffusione della malattia è la probabilità di infezione $p$. Nel modello molto semplice
    abbiamo realizzato, l'infezione si diffonde molto probabilmente in tutto il reticolo anche con un livello di infettività
    relativamente basso: Ciò che possiamo osservare è quindi una diversa velocità di propagazione dell'epidemia a seconda
    di questo parametro.
    \begin{figure}[H]
        \centering
        \includegraphics[width=0.6\linewidth]{../images/sir_infectivities}
        \caption{Curve di diffusione di più simulazioni al variare della probabilità di infezione.}
        \label{fig:sir_infectivities}
    \end{figure}
    Possiamo notare in Figura \ref{fig:sir_infectivities} che all'aumentare della probabilità di infezione, la curva di
    individui refrattari, ovvero infettati e guariti, si sposta verso sinistra, indicando una maggiore velocità di
    propagazione dell'epidemia.\\
    Un ulteriore fattore critico oltre ai due già descritti è dato dal numero di vicini scelti per ogni individuo. In una tale griglia
    è possibile scegliere di considerare solo i primi 4 vicini (alto, basso, destra, sinistra) oppure tutte le 8 celle adiacenti.
    In questo ultimo caso ci aspettiamo che la soglia di infettività sia più bassa, in quanto la probabilità di trovarsi
    vicino a un individuo suscettibile è maggiore. Possiamo infatti notare in Figura \ref{fig:sir_neighbors} che la curva
    relativa agli 8 vicini è più ripida rispetto a quella relativa agli 4 vicini.
    \begin{figure}[H]
        \centering
        \includegraphics[width=0.6\linewidth]{../images/sir_neighbors}
        \caption{Frazione di individui refrattari relativi a una simulazione con $p=0.8$}
        \label{fig:sir_neighbors}
    \end{figure}

    Questo modello molto semplice non tiene ovviamente conto di caratteristiche molto più complesse della malattia e della
    popolazione, come la mobilità degli individui, la densità di popolazione, la distribuzione dell'età e la presenza di
    misure di mitigazione come il distanziamento sociale e la quarantena. Tuttavia, fornisce una base teorica per
    comprendere la diffusione delle malattie infettive e come le strategie di controllo possono influenzare questa diffusione.

    \section{SIS}
    Il modello SIR assume che gli individui guariti acquisiscano immunità permanente alla malattia.
    Tuttavia, se ciò non valesse, esistono modelli alternativi che considerano una certa probabilità di reinfezione.
    Il caso all'estremo opposto è infatti il modello SIS (Susceptible-Infectious-Susceptible), in cui gli individui
    guariti tornano a essere suscettibili in egual maniera rispetto ai sani.
    Questo modello è particolarmente interessante per le malattie in cui l'immunità non è permanente, come l'influenza
    stagionale o il raffreddore comune.\\
    Anche per questo modello possiamo eseguire delle simulazioni analoghe alle precedenti, utilizzando lo stesso reticolo ma
    assumendo che gli individui guariti tornino a essere suscettibili.
    \begin{figure}[H]
        \centering
        \begin{subfigure}[b]{0.49\linewidth}
         \centering
         \includegraphics[width=\linewidth]{../images/sis_image1}
     \end{subfigure}
        \hfill
        \begin{subfigure}[b]{0.49\linewidth}
         \centering
         \includegraphics[width=\linewidth]{../images/sis_image2}
     \end{subfigure}
     \caption{Simulazione del modello SIS con infettività $p=0.2$ e tasso di guarigione $g=0.5$.}
     \label{fig:sis_image}
    \end{figure}
    In questo caso le curve saranno diverse da quelle del modello SIR, poiché non abbiamo una frazione di refrattari.
    A seconda del grado di infettività e di guarigione è possibile osservare andamenti diversi tra loro, ma che hanno
    tutti la caratteristica che avere un andamento asintotico nel lungo termine simile a una saturazione. In Figura
    \ref{fig:sis_curves} sono mostrati i risultati ottenuti con diversi valori dei parametri di infettività e di guarigione.
    \begin{figure}[H]
        \centering
        \begin{subfigure}[b]{0.49\linewidth}
         \centering
         \includegraphics[width=\linewidth]{../images/SIS_curves}
            \caption{Simulazione SIS con $p=0.6$ e $g=0.2$}
     \end{subfigure}
        \hfill
        \begin{subfigure}[b]{0.49\linewidth}
         \centering
         \includegraphics[width=\linewidth]{../images/SIS_curves1}
            \caption{Simulazione SIS con $p=0.2$ e $g=0.5$}
     \end{subfigure}
     \caption{Curve empiriche di due simulazioni del modello SIS.}
     \label{fig:sis_curves}
    \end{figure}

    \section{Percolazione}
    Il modello SIR è analogo alla crescita di una componente di percolazione standard connesso al sito di partenza.
    La percolazione standard è un modello fondamentale della teoria dei sistemi complessi che studia la formazione di un
    percorso connesso attraverso una rete in funzione di una certa probabilità di occupazione dei siti nella rete stessa.
    Il concetto chiave della percolazione standard è la ricerca di un percorso connesso o \textit{cluster} che attraversa
    la rete da un lato all'altro. Essa studia quindi come la presenza di occupazioni casuali nei siti influenzi la
    probabilità di avere un percorso connesso che attraversi l'intera rete.
    La percolazione può essere utilizzata per comprendere fenomeni come la conduttività elettrica dei materiali disordinati,
    la diffusione di fluidi attraverso mezzi porosi e la propagazione di informazioni o malattie attraverso reti sociali.\\
    In questo senso, dunque, è possibile costruire un modello di percolazione che rappresenti la diffusione dell'epidemia
    all'interno di una popolazione. In questo modello, i siti della rete rappresentano gli individui della popolazione e
    i legami tra i siti rappresentano i contatti sociali tra gli individui. La probabilità di occupazione di un sito
    dipende da dei numeri casuali e corrisponde alla probabilità che l'individuo sia infetto. Ciò che cerchiamo è dunque
    un percorso connesso di siti infetti che attraversi interamente la rete, indicando una diffusione dell'epidemia in
    tutta la popolazione.\\
    Vediamo più in dettaglio il modello fin qui descritto, in particolare riferendoci a un sottoinsieme di problemi di
    percolazione detto \textit{Site Percolation}. Ad ogni sito può essere associato un valore binario $x_i$, che
    rappresenta lo stato del sito: $x_i=1$ se il sito è infetto, $x_i=0$ altrimenti. La probabilità di infezione di un
    sito dipende dal parametro di infettività $p$ e da un numero casuale $r_i$ associato al sito. Inseriamo infine una
    variabile temporale $t$ che ci servirà per descrivere l'evoluzione dell'epidemia. Possiamo definire l'evoluzione dello
    stato del sito $i$ al tempo $t$ come:
    \[x_i(t)=x_i(t-1)\lor[r_i < p]\left[\sum_{j} a_{ij}x_j(t-1)>0 \right]\]
    dove $a_{ij}$ è la matrice di adiacenza del grafo, ovvero $a_{ij}=1$ se i siti $i$ e $j$ sono vicini, $0$ altrimenti;
    mentre $[\cdot]$ è la funzione di verità che restituisce $1$ se la condizione tra parentesi è vera, $0$ altrimenti.
    Notiamo che i numeri casuali $r_i$ rimangono costanti nel tempo, dunque per un certo valore di $p$ e una data
    configurazione iniziale i cluster crescono fino a un certo punto e poi si arrestano.\\
    \begin{figure}[H]
        \centering
        \begin{subfigure}[b]{0.49\linewidth}
         \centering
         \includegraphics[width=\linewidth]{../images/percolation_image1}
     \end{subfigure}
        \hfill
        \begin{subfigure}[b]{0.49\linewidth}
         \centering
         \includegraphics[width=\linewidth]{../images/percolation_image2}
     \end{subfigure}
     \caption{Simulazioni del modello di percolazione con diversi gradi di infettività.}
     \label{fig:percolation_image}
    \end{figure}
    In Figura \ref{fig:percolation_image} sono mostrate due simulazioni del modello di percolazione. Nel reticolo sulla
    sinistra, con una probabilità di infezione $p=0.5$, non è possibile trovare un percorso connesso di siti infetti che
    attraversi l'intera rete. Al contrario, nel reticolo sulla destra, con $p=0.6$, si può osservare un caso di percolazione,
    ovvero è presente un percorso connesso che attraversa tutto la griglia. Questo indica che l'epidemia si è diffusa in
    tutta la popolazione.\\
    Può dunque essere interessante cercare la criticità dell'infezione, ovvero il valore di soglia che separa i due
    regimi di percolazione e non percolazione.
    \subsection{Criticità}
    La soglia di criticità, che indichiamo con $p_c$, può essere stimata attraverso diversi metodi; in questo studio
    analizzeremo prima di tutto il metodo più semplice, ovvero quello di eseguire molte simulazioni modificando ogni
    volta il parametro di infettività.
    Dopodiché utilizzeremo un metodo più sofisticato, basato sulla formulazione SOC (Self-Organized Criticality), che
    permette di ottenere una stima affidabile eseguendo una sola simulazione.\\
    Vediamo in questa sezione i risultati ottenuti con il metodo più semplice. Costruiamo più volte un reticolo
    analogo ai precedenti, in cui si inizializzano come infetti tutti i siti sulla prima riga. Ad ogni step, quindi,
    ogni sito sano sarà infettato da un vicino infetto con una certa probabilità $p$. Modifichiamo il valore di $p$ a
    ogni simulazione in modo da ottenere una curva di percolazione.
    \begin{figure}[H]
        \centering
        \includegraphics[width=0.6\linewidth]{../images/percolation_threshold}
        \caption{Percentuale di percolazioni rispetto al numero di simulazioni, al variare del valore di infettività.
                    La riga verticale tratteggiata indica il valore di soglia teorico $p_c=0.5927$. I risultati sono stati ottenuti eseguendo 10 simulazioni per
                    ogni diverso valore di $p$, dopodiché è stata calcolata la frazione di percolazioni su tale numero.}
        \label{fig:percolation_critical}
    \end{figure}
    Come si può osservare in Figura \ref{fig:percolation_critical}, la curva assume una caratteristica forma a S e rispetta
    il valore di soglia teorico per un reticolo 2D pari a $p_c = 0.5927$.
    Nonostante si riesca a trovare una stima affidabile di criticità, questo metodo è computazionalmente molto costoso,
    può essere vantaggioso utilizzare metodi più efficienti come quello che mostreremo di seguito.
    \subsection{Percolazione Auto-Organizzata (SOC)}
    La formulazione SOC (Self-Organized Criticality) è utilizzata in diversi contesti e può essere applicata nel nostro
    caso per ottenere una stima affidabile del valore di soglia di percolazione. Ciò che è necessario inizialmente è
    mappare il nostro problema di percolazione in un SOC equivalente e, per farlo, possiamo utilizzare un metodo chiamato
    \textit{Invasion Percolation}.\\
    Vediamo prima la formulazione matematica del problema, dopodiché passiamo alla progettazione pratica del modello.
    Supponiamo di avere un reticolo 2D e una certa probabilità di infezione $p$. Come nel caso precedente, a ogni sito
    associamo uno stato binario $x_i$ e un numero casuale $r_i$. Definiamo inoltre una variabile $t$ che rappresenta il
    i diversi istanti di tempo. Ricordiamo che l'evoluzione dello stato del sito $i$ al tempo $t$ è data da:
    \[x_i(t)=x_i(t-1)\lor[r_i < p]\left[\sum_{j} a_{ij}x_j(t-1)>0 \right].\]
    Riscriviamo per semplicità il secondo termine:
    \[x_i(t)=x_i(t-1)\lor\left([r_i < p]\land \left(\bigvee\limits_{j:a_{ij}=1} x_j(t)\right)\right).\]
    Possiamo sostituire lo stato $x_i(t)$ con $[p>p_i(t)]$ poiché il sito si bagna (si infetta) solo se la probabilità
    di infezione è maggiore della sua soglia critica $p_i(t)$. Allora l'espressione diventa:
    \[[p_i(t+1) < p]=[p_i(t)<p] \lor \left( [r_i<p]\left( \bigvee\limits_{j:a_{ij}=1}[p_j(t)<p] \right) \right).\]
    In generale, sappiamo che valgono $[a<p]\land[b<p]=[\max(a,b)<p]$ e $[a<p]\lor[b<p]=[\min(a,b)<p]$. Allora la nostra
    espressione diventa:
    \[[p_i(t+1)<p]=\left[\min\left( p_i(t), \max\left( r_i, \min_{j:a{ij}=1}\left( p_j(t)\right) \right) \right)<p\right].\]
    A questo punto diventa semplice estrarre un'equazione per la soglia di criticità $p_i$:
    \[p_i(t+1)=\min\left( p_i(t), \max\left( r_i, \min_{j:a{ij}=1}\left( p_j(t)\right) \right) \right).\]
    Iterando finché non si raggiunge un fenomeno di percolazione, possiamo ottenere una distribuzione di $p_i$ che tende a $p_c$.\\
    Vediamo adesso come implementare questo metodo all'interno del nostro reticolo. Si può pensare di inizializzare tutti
    i siti della prima riga come infetti. La probabilità di infezione $p$ può essere inizializzata a zero e poi aumentata
    ad ogni step dell'algoritmo. Ad ogni iterazione, si cerca, tra tutti i vicini degli infetti, il sito $i$ con la soglia di
    criticità $p_i$ minore; si aumenta dunque l'infettività a $p=p_i$ e si infetta tale sito, dopodiché di bagnano anche
    tutti i nodi adiacenti $j$ che hanno $p_j<p$. Quando non ci sono più nodi infettabili, si esegue una nuova ricerca del minimo
    e il processo viene ripetuto finché non si raggiunge uno stato di percolazione del reticolo.\\
    \begin{figure}[H]
        \centering
        \begin{subfigure}[b]{0.49\linewidth}
         \centering
         \includegraphics[width=\linewidth]{../images/invasionpercolation_image}
            \caption{Immagine di una simulazione sul reticolo.}
     \end{subfigure}
        \hfill
        \begin{subfigure}[b]{0.49\linewidth}
         \centering
         \includegraphics[width=\linewidth]{../images/invasionpercolation}
            \caption{Stime della soglia di criticità ottenute con più simulazioni.}
     \end{subfigure}
     \caption{Simulazioni del modello di percolazione SOC per la ricerca della soglia di criticità.}
     \label{fig:invasion_percolation}
    \end{figure}
    Con il metodo descritto è possibile ottenere una stima abbastanza affidabile del valore di soglia di percolazione
    con un'unica simulazione. In Figura \ref{fig:invasion_percolation} (destra) sono mostrati i risultati ottenuti a
    partire da una sola simulazione fino a 20. Notiamo che eseguendo il processo più volte si ottiene una stima sempre
    più precisa, ma già inizialmente il valore ottenuto è molto vicino a quello teorico di $p_c = 0.5927$.\\
    Come si può facilmente intuire, il metodo SOC è molto più efficiente rispetto a quello delle molte simulazioni;
    anche eseguendolo più volte infatti, il tempo di calcolo è molto inferiore rispetto al metodo precedente.

    \section{Conclusioni}
    Nell'ambito di questo studio, abbiamo esaminato una serie di modelli epidemiologici, a partire dal più semplice modello
    Forest Fire fino ad arrivare al meccanismo di percolazione. Attraverso varie simulazioni abbiamo descritto quelli che
    sono i tipici andamenti delle curve di diffusione di un'epidemia all'interno di una popolazione. Abbiamo inoltre
    esaminato come variabili come la densità di popolazione, i contatti tra gli individui e le strategie di mitigazione
    possano influenzare la diffusione dell'epidemia. Infine, abbiamo spostato la nostra attenzione sul calcolo della soglia
    di criticità di un'epidemia, che abbiamo calcolato attraverso un modello di percolazione auto-organizzato.
    Tutto questo ci ha permesso di ottenere una visione più chiara di come i modelli epidemiologici possano essere utilizzati
    per comprendere e gestire fenomeni epidemici complessi.\\
    Tutti le simulazioni sono state eseguite utilizzando un semplice reticolo 2D, ma è possibile estendere questi modelli
    a modelli di popolazione più complessi, ad esempio basandoci sui grafi di tipo Scale-Free o Small-World, che potrebbero
    fornire una rappresentazione più realistica delle interazioni sociali umane.

    \begin{thebibliography}{9}
\bibitem{review}
F. Bagnoli, E. Bellini, E. Massaro, R. Retchman (2019) \emph{Percolation and Internet Science}

\bibitem{risk_perception}
E. Massaro, F. Bagnoli (2014) \emph{Epidemic spreading and risk perception in multiplex networks: a self-organized percolation method}

\bibitem{epidemic_threshold}
F. Bagnoli, E. Bellini, E. Massaro (2018) \emph{A Self-Organized Percolation Method for Computing the Epidemic Threshold in Computer Networks}

\end{thebibliography}

\end{document}
